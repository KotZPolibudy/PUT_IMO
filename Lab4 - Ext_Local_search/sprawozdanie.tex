\documentclass[11pt]{article}

\usepackage{amsmath}
\usepackage[a4paper, margin=0.5in]{geometry}
\usepackage{graphicx} % daj an 1in jak chcesz normalniejszy margines, ale kod mi się w linii nie mieści :P
\usepackage[utf8]{inputenc}
\usepackage[T1]{fontenc}
\usepackage[polish]{babel}
\usepackage{float}
\usepackage{hyperref}
\usepackage{cleveref}
\usepackage{subfigure}

\title{Zadanie 4. Rozszerzenia lokalnego przeszukiwania}
\author{Oskar Kiliańczyk 151863 \& Wojciech Kot 151879}
\date{}

\begin{document}

\maketitle
\newpage

\section{Opis zadania}\label{sec:opis-zadania}

Celem eksperymentu jest poprawa i rozszerzenie lokalnego przeszukiwania o trzy nowe metody:\\
- MSLS - Multiple Start local search\\
- ILS - Iterated local search\\
- LNS - Large neighborhood search\\
Porównujemy wersję bazową, znaną z poprzedniego eksperymentu z tymi trzema rozszerzeniami.\\


\section{Opisy algorytmów}\label{sec:opisy-alg}

\subsection{Multiple Start Local Search (MSLS)}\label{subsec:msls}
\begin{enumerate}
    \item Zainicjuj zmienną obecnie najlepszego kosztu jako MAXINT
    \item dla i=1,2,\...ile\_iteracji
    \begin{enumerate}
        \item Wygeneruj losowe rozwiązanie początkowe
        \item Wykonaj lokalne przeszukiwanie
        \item Jeśli obecnie znalezione rozwiązanie ma mniejszy koszt niż obecnie najlepszy
        \begin{enumerate}
            \item zapisz obecnie znalezione rozwiazanie jako najlepsze
            \item zapisz koszt tego rozwiazania jako obecnie najlepszy
        \end{enumerate}
    \end{enumerate}
    \item Zwróć obecnie najlepsze rozwiązanie
\end{enumerate}


\subsection{Iterated Local Search (ILS)}\label{subsec:ils}

Główny algorytm:
\begin{enumerate}
    \item Wygeneruj losowe rozwiązanie początkowe
    \item Wykonaj lokalne przeszukiwanie
    \item zapisz obecnie znalezione rozwiazanie jako najlepsze
    \item zapisz koszt tego rozwiazania jako obecnie najlepszy
    \item Dopóki czas wykonywania < limit\_czasu:
        \begin{enumerate}
            \item Wykonaj funkcję Perturbacji* na obecnie najlepszym cyklu
            \item Jeśli obecnie znalezione rozwiązanie ma mniejszy koszt niż obecnie najlepszy
                \begin{enumerate}
                    \item zapisz obecnie znalezione rozwiazanie jako najlepsze
                    \item zapisz koszt tego rozwiazania jako obecnie najlepszy
                \end{enumerate}
        \end{enumerate}
\end{enumerate}

Funkcja Perturbacji:
\begin{enumerate}
    \item Parametry: cykl1, cykl2, ile\_zmian\_wierzcholkow, ile\_zmian\_krawedzi
    \item dla i=1,2,\...,ile\_zmian\_wierzcholkow:
        \begin{enumerate}
            \item Zamień losowe wierzchołki między cyklami
        \end{enumerate}

    \item dla i=1,2,\...,ile\_zmian\_krawedzi:
        \begin{enumerate}
            \item Zamień losowo wybrane krawędzi w pierwszym cyklu
            \item Zamień losowo wybrane krawędzi w drugim cyklu
        \end{enumerate}
    \item Zwróć cykl1, cykl2
\end{enumerate}


\subsection{Large neighborhood Search (LNS)}\label{subsec:lns}


Główny algorytm:
\begin{enumerate}
    \item Wygeneruj losowe rozwiązanie początkowe
    \item Wykonaj lokalne przeszukiwanie
    \item zapisz obecnie znalezione rozwiazanie jako najlepsze
    \item zapisz koszt tego rozwiazania jako obecnie najlepszy
    \item Dopóki czas wykonywania < limit\_czasu:
        \begin{enumerate}
            \item Wykonaj funkcję destroy\_repair na obecnie najlepszym cyklu
            \item Jeśli ustawiona jest flaga wykonaj\_LS:
            \begin{enumerate}
                \item Wykonaj lokalne przeszukiwanie na obecnym rozwiązaniu
            \end{enumerate}
            \item Oblicz koszt obecnego rozwiązania
            \item Jeśli obecnie znalezione rozwiązanie ma mniejszy koszt niż obecnie najlepszy
                \begin{enumerate}
                    \item zapisz obecnie znalezione rozwiazanie jako najlepsze
                    \item zapisz koszt tego rozwiazania jako obecnie najlepszy
                \end{enumerate}
        \end{enumerate}
    \item Zwróć najlepsze rozwiązanie
\end{enumerate}


Funkcja destroy\_repair:
\begin{enumerate}

\end{enumerate}


\section{Wyniki}\label{sec:wyniki}

\subsection{Tabela wynikowa}\label{subsec:tabela-wynikowa}

TODO

\subsection{Wizualizacja wyników}\label{subsec:wizualizacja-wynikow}

TODO


\section{Wnioski i analiza wyników}\label{sec:wnioski}

TODO

\section{Link do repozytorium}\label{sec:link-do-repo}
Kod źródłowy w repozytorium GitHub dostępny pod linkiem: \\
\href{https://github.com/KotZPolibudy/PUT_IMO/tree/main/Lab3%20-%20Local_augmented}{Repozytorium}.

\end{document}
