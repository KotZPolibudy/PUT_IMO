\documentclass[11pt]{article}

\usepackage{amsmath}
\usepackage[a4paper, margin=0.5in]{geometry}
\usepackage{graphicx} % daj an 1in jak chcesz normalniejszy margines, ale kod mi się w linii nie mieści :P
\usepackage[utf8]{inputenc}
\usepackage[T1]{fontenc}
\usepackage[polish]{babel}
\usepackage{float}
\usepackage{hyperref}
\usepackage{cleveref}
\usepackage{subfigure}

\title{Zadanie 3. Wykorzystanie ocen ruchów z poprzednich iteracji i ruchów
kandydackich w lokalnym przeszukiwaniu}
\author{Oskar Kiliańczyk 151863 \& Wojciech Kot 151879}
\date{}

\begin{document}

\maketitle
\newpage

\section{Opis zadania}\label{sec:opis-zadania}

Celem eksperymentu jest poprawa efektywności czasowej lokalnego przeszukiwania w wersji stromej, wykorzystującego najlepsze sąsiedztwo z poprzedniego zadania.
Porównujemy wersję bazową z dwiema modyfikacjami: uporządkowaną listą ruchów oraz ruchem kandydackim.
Każdy algorytm uruchamiany jest 100 razy na każdej instancji, startując z losowych rozwiązań.
Dla porównania uwzględniamy również najlepszą heurystykę konstrukcyjną z zadania 1.

\section{Opisy algorytmów}\label{sec:opisy-alg}

# todo opisy
# todo pseudokody

\section{Wyniki}\label{sec:wyniki}

\subsection{Tabela wynikowa}\label{subsec:tabela-wynikowa}

\subsection{Wizualizacja wyników}\label{subsec:wizualizacja-wynikow}

\section{Wnioski i analiza wyników}\label{sec:wnioski}


\section{Link do repozytorium}\label{sec:link-do-repo}
Kod źródłowy w repozytorium GitHub dostępny pod linkiem: \\
\href{https://github.com/KotZPolibudy/PUT_IMO/tree/main/Lab3%20-%20Local_augmented}{Repozytorium}.

\end{document}
