% Preamble
\documentclass[11pt]{article}

% Packages
\usepackage{amsmath}

\title{Zadanie 1. Heurystyki konstrukcyjne}
\author{Oskar Kiliańczyk 151863 \& Wojciech Kot 151876}
\date{}

% Document
\begin{document}

\maketitle
\newpage

\section{Opis zadania}\label{sec:opis-zadania}

Podczas zajęć rozważamy zmodyfikowany problem komiwojażera.
Początkowo, obliczamy macierz odległości pomiędzy miastami.
Obliczona macierz odległości między wierzchołkami grafu będzie podstawą dla każdego algorytmu,
a celem jest wyznaczenie dwóch rozłącznych zamkniętych ścieżek (cykli), z których każda zawiera 50\% wierzchołków.
Jeśli liczba wierzchołków jest nieparzysta, pierwsza ścieżka zawiera jeden wierzchołek więcej.
Kryterium optymalizacji jest minimalizacja łącznej długości obu cykli.

Rozważane instancje problemu pochodzą z biblioteki TSPLib, w szczególności kroa200 oraz krob200.
Są to instancje dwuwymiarowe euklidesowe, w których każdemu wierzchołkowi przypisane są współrzędne na płaszczyźnie.
Odległość między wierzchołkami liczona jest jako odległość euklidesowa i zaokrąglana do najbliższej liczby całkowitej.
W implementacji wykorzystywana będzie wyłącznie macierz odległości, co zapewnia możliwość zastosowania kodu do innych instancji problemu.

\section{Zaimplementowane algorytmy}\label{sec:zaimplementowane-algorytmy}

bbbbb

\section{Wyniki eksperymentu obliczeniowego}\label{sec:wyniki-eksperymenty-obliczeniowego}

cccccc i wizualizacje

\section{Wnioski}\label{sec:wnioski}

\section{Link do repo}\label{sec:link-do-repo}

\end{document}
