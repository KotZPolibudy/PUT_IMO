% Preamble
\documentclass[11pt]{article}

% Packages
\usepackage{amsmath}

\title{Zadanie 1. Heurystyki konstrukcyjne}
\author{Oskar Kiliańczyk 151863 \& Wojciech Kot 151876}
\date{}

% Document
\begin{document}

\maketitle
\newpage

\section{Opis zadania}\label{sec:opis-zadania}

Podczas zajęć rozważamy zmodyfikowany problem komiwojażera.
Początkowo, obliczamy macierz odległości pomiędzy danymi miastami.
Obliczona macierz odległości między wierzchołkami grafu będzie podstawą dla każdego algorytmu,
a celem jest wyznaczenie dwóch rozłącznych zamkniętych ścieżek (cykli), z których każda zawiera 50\% wierzchołków.
Jeśli liczba wierzchołków jest nieparzysta, jedna ścieżka zawiera jeden wierzchołek więcej.
Kryterium optymalizacji jest minimalizacja łącznej długości obu cykli.

Rozważane instancje problemu pochodzą z biblioteki TSPLib, a są to kroa200 oraz krob200.
Są to instancje dwuwymiarowe euklidesowe, w których każdemu wierzchołkowi przypisane są współrzędne na płaszczyźnie.
Odległość między wierzchołkami liczona jest jako odległość euklidesowa, zaokrąglana do najbliższej liczby całkowitej.
W implementacji algorytmów wykorzystywana będzie wyłącznie macierz odległości, co zapewnia możliwość zastosowania kodu do innych instancji problemu.

\section{Zaimplementowane algorytmy}\label{sec:zaimplementowane-algorytmy}

\subsection{Algorytm zachłanny - metoda najbliższego sąsiada}\label{subsec:algorytm-zachanny---metoda-najblizszego-sasiada}

Algorytm ten wykorzystuje funkcję znajdującą najbliższego sąsiada dla danego wierzchołka (miasta)
Działa ona w następujący sposób: \\
dla każdego miasta, sprawdza czy zostało już odwiedzone \\
jeśli nie, to sprawdza czy dystans jest mniejszy od dystansu z danego miasta do obecnie zapamiętanego jako najbliższe \\
jeśli jest bliższe niż obecnie pamiętane jako najbliższe, zapamiętuje je jako najbliższe \\
jeśli nie ma żadnego miasta obecnie pamiętanego jako najbliższe, przypisuje to miasto \\

Główny algorytm natomiast, wygląda następująco: \\

\begin{verbatim}
    Przydziela wierzchołki startowe do cykli pierwszego i drugiego
    Tworzy tablicę indeksów miast, zaznaczając wszystkie poza startowymi jako nieodwiedzone
    dopóki istnieją jakieś nieodwiedzone miasta, powtarza następujące 4 kroki:
        Znajduje najbliższego nieodwiedzonego sąsiada do ostatniego wierzchołka cyklu 1.
        dodaje go do cyklu1 oraz zapisuje w tablicy jako odwiedzony
        Znajduje najbliższego nieodwiedzonego sąsiada do ostatniego wierzchołka cyklu 2.
        dodaje go do cyklu2 oraz zapisuje w tablicy jako odwiedzony
    Kiedy już nie ma żadnych nieodwiedzonych miast, dopisuje na koniec cykli ich wierzchołki startowe
    Zwraca oba cykle jako znalezione ścieżki

\end{verbatim}

\subsection{Algorytm zachłanny - metoda rozbudowy cyklu}\label{subsec:algorytm-zachanny---metoda-rozbudowy-cyklu}

\subsection{Algorytm z żalem - metoda rozbudowy cyklu}\label{subsec:algorytm-z-zalem---metoda-rozbudowy-cyklu}

\subsection{Algorytm z żalem - metoda rozbudowy cyklu z żalem ważonym}\label{subsec:algorytm-z-zalem---metoda-rozbudowy-cyklu-z-zalem-wazonym}


\section{Wyniki eksperymentu obliczeniowego}\label{sec:wyniki-eksperymenty-obliczeniowego}

cccccc i wizualizacje

\section{Wnioski}\label{sec:wnioski}

\section{Link do repo}\label{sec:link-do-repo}

\end{document}
